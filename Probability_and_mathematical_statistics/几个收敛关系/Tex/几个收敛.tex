\documentclass[UTF8,12pt]{ctexart}
%加载包
\usepackage{geometry}%排版
%a4版面,页边距1英寸,showframe 增加边框的参数。
% 设置为A4纸,边距适中模式(永中office)
\geometry{%
	width = 210mm,%
	height = 297mm,
	left = 19.1mm,%
	right = 19.1mm,%
	top = 25.4mm,%
	bottom = 25.4mm%
}
%\hyphenpenalty = 1000% 断字设置,值越大,断字越少。
%\setlength{\parindent}{2em}% 缩进
%\setlength{\parskip}{0.5ex} % 段间距

\usepackage{amsmath}
\usepackage{amsfonts}
\usepackage{amssymb}%公式
\usepackage{amsthm}%定理环境

%\usepackage{ntheorem}%定理环境,使用这个会使\maketitle版式出问题
\usepackage{bm}%加粗

\usepackage{mathrsfs}
\numberwithin{equation}{section}%对公式以节{section}为基础进行编号.变成(1.1.1)有chapter才有1.1.1,不然只有section是1.1
%\theoremstyle{plain}%定理用latex默认的版式
\newtheorem{thm}{Theorem}[section]
%\theoremstyle{definition}%定义用definition格式
\newtheorem{defn}{Definition}
%\theoremstyle{remark}%用remark格式
\newtheorem{lemma}{Lemma}[section]
\newtheorem{example}{Example}[section]
\newtheorem{proposition}{Proposition}[section]
%\newtheorem{proposition}[thm]{Proposition}%和thm定理一起计数
\usepackage{multirow}%表格列合并宏包,\multirow命令.

\usepackage{tabularx}%表格等宽,\begin{tabularx}命令.

\usepackage{tcolorbox}%盒子
\tcbuselibrary{skins, breakable}% 支持文本框跨页
\usepackage[english]{babel}% 载入美式英语断字模板

\usepackage{graphicx}

\usepackage{float}

\usepackage{listings}   %代码块
\usepackage{xcolor}
\definecolor{codegreen}{rgb}{0,0.6,0}
\definecolor{codegray}{rgb}{0.5,0.5,0.5}
\definecolor{codepurple}{rgb}{0.58,0,0.82}
\definecolor{backcolour}{rgb}{0.95,0.95,0.92}
%设置代码块
\lstdefinestyle{mystyle}{
	backgroundcolor=\color{backcolour},   
	commentstyle=\color{codegreen},
	keywordstyle=\color{magenta},
	numberstyle=\tiny\color{codegray},
	stringstyle=\color{codepurple},
	basicstyle=\ttfamily\footnotesize,
	breakatwhitespace=false,         
	breaklines=true,                 
	captionpos=b,                    
	keepspaces=true,                 
	numbers=left,                    
	numbersep=5pt,                  
	showspaces=false,                
	showstringspaces=false,
	showtabs=false,                  
	tabsize=2
}

\lstset{style=mystyle,
	language=R,                                       % 设置语言
}

\usepackage{appendix}%附录


%标题页
\title{几个常用的收敛}
\author{Renhe W.}
\date{ }
\usepackage{hyperref}%可以生成pdf书签,可以跳转
\hypersetup{
	colorlinks=true,
	linkcolor=black
}%使得目录没有红框

%工具
%使用文本框
%\begin{tcolorbox}[enhanced]	\end{tcolorbox}
%代码框
%{\setmainfont{Courier New Bold}                       %设置代码字体                   
%\begin{lstlisting}
	
%\end{lstlisting}}

%\usepackage{titlesec}
%% 调整 section 标题格式
%\titleformat{\section}
%{\normalfont\large\bfseries}  % 字体设置:小号、加粗
%{\thesection}  % 标签(例如“1.1”)
%{1em}  % 标签和标题文字之间的距离
%{}  % 前缀代码
%[\raggedright]  % 后缀代码(使标题靠左对齐)
%
%% 调整 section 标题间距
%\titlespacing{\section}
%{0pt}  % 左边距
%{*1}   % 标题前的间距
%{*1}   % 标题后的间距



%文章开始部分
\begin{document}
	\maketitle
	%\tableofcontents
	%\newpage
	\section{Definitions}
	统计学中的几个收敛的定义如下:
	\begin{defn}[几乎必然收敛 (Almost Surely Convergence, a.s.)]
		序列 $\left\{X_n\right\}$ 几乎必然收敛到 $X$ ,记为 $X_n \stackrel{\text { a.s. }}{\longrightarrow} X$ ,如果:
		$$
		P\left(\lim _{n \rightarrow \infty} X_n=X\right)=1.
		$$
	\end{defn}
	
	\begin{defn}[依概率收敛 (Convergence in Probability)]
		序列 $\left\{X_n\right\}$ 依概率收敛到 $X$ ,记为 $X_n \stackrel{p}{\longrightarrow} X$ ,如果对任意 $\epsilon>0$ ,有:
		$$
		\lim _{n \rightarrow \infty} P\left(\left|X_n-X\right|>\epsilon\right)=0.
		$$
	\end{defn}
	
	\begin{defn}[依分布收敛 (Convergence in Distribution)]
		序列 $\left\{X_n\right\}$ 依分布收敛到 $X$ ,记为 $X_n \stackrel{d}{\longrightarrow} X$ ,如果对于任意连续有界函数 $g$ ,有:
		$$
		\lim _{n \rightarrow \infty} E\left[g\left(X_n\right)\right]=E[g(X)]
		$$
		或者等价地,对所有 $x$ 在哪里分布函数 $F$ 是连续的,有:
		$$
		\lim _{n \rightarrow \infty} F_n(x)=F(x).
		$$
	\end{defn}
	
	\begin{tcolorbox}[enhanced,title={三者关系}]
		
		\begin{enumerate}
			\item 几乎必然收敛 $\Rightarrow$ 依概率收敛\label{1}\\
			如果 $X_n \stackrel{\text { a.s. }}{\longrightarrow} X$ ,则 $X_n \stackrel{p}{\longrightarrow} X$. 这个结果可以通过Egorov定理来证明.
			\item  依概率收敛 $\Rightarrow$ 依分布收敛\label{2}\\
			如果 $X_n \stackrel{p}{\longrightarrow} X$ ,则 $X_n \stackrel{d}{\longrightarrow} X$. 这可以从依概率收敛的定义直接推导出来.
			\item 几乎必然收敛 $\nRightarrow$ 依分布收敛\\
			几乎必然收敛不一定导致依分布收敛,除非 $X_n$ 和 $X$ 定义在同一个概率空间中。当它们在 同一概率空间时,由\ref{1}和\ref{2}的关系,我们可以说几乎必然收敛导致依分布收敛.		
		\end{enumerate}		
	\end{tcolorbox}
	
	\section{Proofs}
	\subsection{证明 1: 几乎必然收敛 $\Rightarrow$ 依概率收敛}
	\begin{thm}
		如果 $X_n \stackrel{\text { a.s. }}{\longrightarrow} X$ ,则 $X_n \stackrel{p}{\longrightarrow} X$.
	\end{thm}
	\begin{proof}
		假设 $X_n \stackrel{\text { a.s. }}{\longrightarrow} X$ ,我们要证明对于任意的 $\epsilon>0$ ,有:
		$$
		\lim _{n \rightarrow \infty} P\left(\left|X_n-X\right|>\epsilon\right)=0
		$$
		由于 $X_n$ 几乎必然收敛到 $X$ ,我们有:
		$$
		P\left(\lim _{n \rightarrow \infty} X_n=X\right)=1,
		$$
		这意味着,对于任意的 $\epsilon>0$ ,事件 $\left\{\left|X_n-X\right|>\epsilon\right.$ i.o. $\}$ (i.o. 是指无穷次) 的概率是 0 , 即:
		$$
		P\left(\left|X_n-X\right|>\epsilon \text { i.o. }\right)=0,
		$$
		由 Borel-Cantelli 引理,如果 $P\left(\left|X_n-X\right|>\epsilon\right.$ i.o. $)=0$ ,那么 $P\left(\left|X_n-X\right|>\right.$ $\epsilon$ finitely many) $=1$ ,即只有有限个 $n$ 使得 $\left|X_n-X\right|>\epsilon$ 。这意味着对于足够大的 $n$ , 我们有:
		$$
		P\left(\left|X_n-X\right|>\epsilon\right)=0,
		$$
		所以 $X_n \stackrel{p}{\longrightarrow} X$.
	\end{proof}
	
	\begin{thm}[反推条件]
		如果$X_n \stackrel{p}{\longrightarrow} X$且$\{X_n\}$是单调的,则$X_n \stackrel{\text { a.s. }}{\longrightarrow} X$.
	\end{thm}
	
	\subsection{证明 2: 依概率收敛 $\Rightarrow$ 依分布收敛}
	
	
	\begin{proposition}[准备知识] 这里介绍几个要用的知识:\\
		$P(X+Y \leqslant a+b) \leq P(X \leqslant a)+P(Y \leqslant b)$\\
		$P(X+Y \leq a+b) \geqslant P(X \leq a$ 且 $Y \leqslant b)$\\
		$P(A B) \geqslant P(A)-P(\bar{B})$
		$$
		\begin{aligned}
			& P(A B)=P(A \bar{B})=P(A-\bar{B}) \\
			&=P(A)-P(A \bar{B}) \\
			& P(A \bar{B}) \leq P(\bar{B}) P(A B) \geq P(A)-P(\bar{B})
		\end{aligned}
		$$	
	\end{proposition}
	
	\begin{thm}
		如果 $X_n \stackrel{p}{\longrightarrow} X$ ,则 $X_n \stackrel{d}{\longrightarrow} X$.
	\end{thm}
	
	\begin{proof}
		假设 $X_n \stackrel{p}{\longrightarrow} X$ ,我们要证明 $X_n \stackrel{d}{\longrightarrow} X$ ,即对于任意的连续点 $x$ ,我们有:
		$$
		\lim _{n \rightarrow \infty} F_n(x)=F(x),
		$$
		其中 $F_n$ 和 $F$ 分别是 $X_n$ 和 $X$ 的分布函数. 我们可以用依概率收敛的定义来证明这一点. 首先,我们注意到:
		$$
		\left|F_n(x)-F(x)\right|=\left|P\left(X_n \leq x\right)-P(X \leq x)\right| \leq P\left(\left|X_n-X\right|>\epsilon\right)
		$$
		由于 $X_n \stackrel{p}{\longrightarrow} X$ ,我们知道:
		$$
		\lim _{n \rightarrow \infty} P\left(\left|X_n-X\right|>\epsilon\right)=0,
		$$
		这意味着:
		$$
		\lim _{n \rightarrow \infty}\left|F_n(x)-F(x)\right|=0,
		$$
		所以 $X_n \stackrel{d}{\longrightarrow} X$.
	\end{proof}

	\begin{thm}[反推条件:退化分布条件下]
		如果 $X_n \stackrel{d}{\longrightarrow} C$,$C$为常数 ,则$X_n \stackrel{p}{\longrightarrow} C$ .
	\end{thm}

	
\end{document}