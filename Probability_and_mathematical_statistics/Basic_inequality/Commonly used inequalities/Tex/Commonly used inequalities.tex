\documentclass[UTF8,12pt]{ctexart}
%加载包
\usepackage{geometry}%排版
%a4版面,页边距1英寸,showframe 增加边框的参数。
% 设置为A4纸,边距适中模式(永中office)
\geometry{%
	width = 210mm,%
	height = 297mm,
	left = 19.1mm,%
	right = 19.1mm,%
	top = 25.4mm,%
	bottom = 25.4mm%
}
%\hyphenpenalty = 1000% 断字设置,值越大,断字越少。
%\setlength{\parindent}{2em}% 缩进
%\setlength{\parskip}{0.5ex} % 段间距

\usepackage{amsmath}
\usepackage{amsfonts}
\usepackage{amssymb}%公式
\usepackage{amsthm}%定理环境

%\usepackage{ntheorem}%定理环境,使用这个会使\maketitle版式出问题
\usepackage{bm}%加粗

\usepackage{mathrsfs}
\numberwithin{equation}{section}%对公式以节{section}为基础进行编号.变成(1.1.1)有chapter才有1.1.1,不然只有section是1.1
%\theoremstyle{plain}%定理用latex默认的版式
\newtheorem{thm}{Theorem}[section]
%\theoremstyle{definition}%定义用definition格式
\newtheorem{defn}{Definition}
%\theoremstyle{remark}%用remark格式
\newtheorem{lemma}[thm]{lemma}
\newtheorem{example}{Example}[section]

\usepackage{multirow}%表格列合并宏包,\multirow命令.

\usepackage{tabularx}%表格等宽,\begin{tabularx}命令.

\usepackage{tcolorbox}%盒子
\tcbuselibrary{skins, breakable}% 支持文本框跨页
\usepackage[english]{babel}% 载入美式英语断字模板

\usepackage{graphicx}

\usepackage{float}

\usepackage{listings}   %代码块
\usepackage{xcolor}
\definecolor{codegreen}{rgb}{0,0.6,0}
\definecolor{codegray}{rgb}{0.5,0.5,0.5}
\definecolor{codepurple}{rgb}{0.58,0,0.82}
\definecolor{backcolour}{rgb}{0.95,0.95,0.92}
%设置代码块
\lstdefinestyle{mystyle}{
	backgroundcolor=\color{backcolour},   
	commentstyle=\color{codegreen},
	keywordstyle=\color{magenta},
	numberstyle=\tiny\color{codegray},
	stringstyle=\color{codepurple},
	basicstyle=\ttfamily\footnotesize,
	breakatwhitespace=false,         
	breaklines=true,                 
	captionpos=b,                    
	keepspaces=true,                 
	numbers=left,                    
	numbersep=5pt,                  
	showspaces=false,                
	showstringspaces=false,
	showtabs=false,                  
	tabsize=2
}

\lstset{style=mystyle,
	language=R,                                       % 设置语言
}

\usepackage{appendix}%附录


%标题页
\title{Commonly used inequalities}
\author{Renhe W.}
\date{ }
\usepackage{hyperref}%可以生成pdf书签,可以跳转
\hypersetup{
	colorlinks=true,
	linkcolor=black
}%使得目录没有红框

%工具
%使用文本框
%\begin{tcolorbox}[enhanced]	\end{tcolorbox}
%代码框
%{\setmainfont{Courier New Bold}                       %设置代码字体                   
%\begin{lstlisting}
	
%\end{lstlisting}}

\usepackage{titlesec}
% 调整 section 标题格式
\titleformat{\section}
{\normalfont\large\bfseries}  % 字体设置:小号、加粗
{\thesection}  % 标签(例如“1.1”)
{1em}  % 标签和标题文字之间的距离
{}  % 前缀代码
[\raggedright]  % 后缀代码(使标题靠左对齐)

% 调整 section 标题间距
\titlespacing{\section}
{0pt}  % 左边距
{*1}   % 标题前的间距
{*1}   % 标题后的间距



%文章开始部分
\begin{document}
	\maketitle
	%\tableofcontents
	%\newpage
	在统计学的渐进理论和证明中,一些经典的不等式被广泛使用,以下是其中一些常用的不等式:
	\section{ Markov 不等式 (Markov's Inequality)}
	设 $X$ 是一个非负的随机变量,对任意 $a>0$ ,有:
	$$
	P(X \geq a) \leq \frac{E[X]}{a}.
	$$
	\section{Chebyshev 不等式 (Chebyshev's Inequality)}
	设 $X$ 是一个具有有限均值 $\mu$ 和方差 $\sigma^2$ 的随机变量,对任意 $k>0$ ,有:
	$$
	P(|X-\mu| \geq k \sigma) \leq \frac{1}{k^2}.
	$$
	\section{Jensen 不等式 (Jensen's Inequality)}
	设 $\phi$ 是一个凸函数,如果 $E[|X|]<\infty$ 并且 $E[|\phi(X)|]<\infty$ ,则:
	$$
	\phi(E[X]) \leq E[\phi(X)].
	$$
	\section{Hölder 不等式 (Hölder's Inequality)}
	设 $p>0$ 和 $q>0$ 满足 $\frac{1}{p}+\frac{1}{q}=1$ ,如果 $X$ 和 $Y$ 是随机变量并且 $E\left[|X|^p\right]<\infty$ 和 $E\left[|Y|^q\right]<\infty$ ,则:
	$$
	|E[X Y]| \leq\left(E\left[|X|^p\right]\right)^{1 / p}\left(E\left[|Y|^q\right]\right)^{1 / q}.
	$$
	\section{ Minkowski 不等式 (Minkowski's Inequality)}
	设 $p \geq 1$ ,如果 $X$ 和 $Y$ 是随机变量且 $E\left[|X|^p\right]<\infty$ 和 $E\left[|Y|^p\right]<\infty$ ,则:
	$$
	\left(E\left[|X+Y|^p\right]\right)^{1 / p} \leq\left(E\left[|X|^p\right]\right)^{1 / p}+\left(E\left[|Y|^p\right]\right)^{1 / p}.
	$$
	\section{Cauchy-Schwarz 不等式 (Cauchy-Schwarz Inequality)} 
	
	如果 $X$ 和 $Y$ 是随机变量且 $E\left[X^2\right]<\infty$ 和 $E\left[Y^2\right]<\infty$ ,则:
	$$
	|E[X Y]|^2 \leq E\left[X^2\right] E\left[Y^2\right].
	$$
	\section{Chernoff 不等式 (Chernoff's Inequality)} 
	
	对于任意 $t>0$ 和随机变量 $X$ ,有:
	$$
	P(X \geq t) \leq \frac{E\left[e^{\lambda X}\right]}{e^{\lambda t}},
	$$
	对任意正的 $\lambda$.

	
\end{document}