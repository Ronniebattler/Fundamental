\documentclass[UTF8,12pt]{ctexart}
%加载包
\usepackage{geometry}%排版
%a4版面,页边距1英寸,showframe 增加边框的参数。
% 设置为A4纸,边距适中模式(永中office)
\geometry{%
	width = 210mm,%
	height = 297mm,
	left = 19.1mm,%
	right = 19.1mm,%
	top = 25.4mm,%
	bottom = 25.4mm%
}
%\hyphenpenalty = 1000% 断字设置,值越大,断字越少。
%\setlength{\parindent}{2em}% 缩进
%\setlength{\parskip}{0.5ex} % 段间距

\usepackage{amsmath}
\usepackage{amsfonts}
\usepackage{amssymb}%公式
\usepackage{amsthm}%定理环境

%\usepackage{ntheorem}%定理环境,使用这个会使\maketitle版式出问题
\usepackage{bm}%加粗

\usepackage{mathrsfs}
\numberwithin{equation}{section}%对公式以节{section}为基础进行编号.变成(1.1.1)有chapter才有1.1.1,不然只有section是1.1
%\theoremstyle{plain}%定理用latex默认的版式
\newtheorem{thm}{Theorem}[section]
%\theoremstyle{definition}%定义用definition格式
\newtheorem{defn}{Definition}
%\theoremstyle{remark}%用remark格式
\newtheorem{lemma}{Lemma}[section]
\newtheorem{example}{Example}[section]
\newtheorem{proposition}{Proposition}[section]

\usepackage{multirow}%表格列合并宏包,\multirow命令.

\usepackage{tabularx}%表格等宽,\begin{tabularx}命令.

\usepackage{tcolorbox}%盒子
\tcbuselibrary{skins, breakable}% 支持文本框跨页
\usepackage[english]{babel}% 载入美式英语断字模板

\usepackage{graphicx}

\usepackage{float}

\usepackage{listings}   %代码块
\usepackage{xcolor}
\definecolor{codegreen}{rgb}{0,0.6,0}
\definecolor{codegray}{rgb}{0.5,0.5,0.5}
\definecolor{codepurple}{rgb}{0.58,0,0.82}
\definecolor{backcolour}{rgb}{0.95,0.95,0.92}
%设置代码块
\lstdefinestyle{mystyle}{
	backgroundcolor=\color{backcolour},   
	commentstyle=\color{codegreen},
	keywordstyle=\color{magenta},
	numberstyle=\tiny\color{codegray},
	stringstyle=\color{codepurple},
	basicstyle=\ttfamily\footnotesize,
	breakatwhitespace=false,         
	breaklines=true,                 
	captionpos=b,                    
	keepspaces=true,                 
	numbers=left,                    
	numbersep=5pt,                  
	showspaces=false,                
	showstringspaces=false,
	showtabs=false,                  
	tabsize=2
}

\lstset{style=mystyle,
	language=R,                                       % 设置语言
}

\usepackage{appendix}%附录


%标题页
\title{Law of Large Numbers}
\author{Renhe W.}
\date{ }
\usepackage{hyperref}%可以生成pdf书签,可以跳转
\hypersetup{
	colorlinks=true,
	linkcolor=black
}%使得目录没有红框

%工具
%使用文本框
%\begin{tcolorbox}[enhanced]	\end{tcolorbox}
%代码框
%{\setmainfont{Courier New Bold}                       %设置代码字体                   
%\begin{lstlisting}
	
%\end{lstlisting}}

%\usepackage{titlesec}
%% 调整 section 标题格式
%\titleformat{\section}
%{\normalfont\large\bfseries}  % 字体设置:小号、加粗
%{\thesection}  % 标签(例如“1.1”)
%{1em}  % 标签和标题文字之间的距离
%{}  % 前缀代码
%[\raggedright]  % 后缀代码(使标题靠左对齐)
%
%% 调整 section 标题间距
%\titlespacing{\section}
%{0pt}  % 左边距
%{*1}   % 标题前的间距
%{*1}   % 标题后的间距



%文章开始部分
\begin{document}
	\maketitle
	%\tableofcontents
	%\newpage
	\section{Definitions and Proofs}
	证明之前,回顾一些引理:
	\begin{lemma}[Lévy 极限定理 (Lévy's Convergence Theorem)]
		如果一个随机变量序列 $\left\{X_n\right\}$ 独立同分布,且存在常数 $E\left[X_n\right]=\mu$和 $\operatorname{Var}\left[X_n\right]=\sigma^2<$ $\infty$ ,那么序列 $\left\{\bar{X}_n\right\}$ ,其中 $\bar{X}_n=\frac{1}{n} \sum_{i=1}^n X_i$,满足
		$$
		\lim _{n \rightarrow \infty} P\left(\frac{\bar{X}_n-\mu}{\sigma / \sqrt{n}} \leq x\right)=\Phi(x),
		$$
		其中 $\Phi(x)$ 是均值为 0 和方差为 1 的正态分布的累积分布函数.
	\end{lemma}
	
	
	
	统计学中常用的几个大数定理的定义如下:
	
	\begin{thm}[伯努利大数定理 (Bernoulli's Law of Large Numbers)]
		对于独立同分布(i.i.d.)的随机变量序列 $\left\{X_n\right\}$ ,每个 $X_n$ 取值于 $\{0,1\}$ ,且 $P\left(X_n=1\right)=$ $p$ 和 $P\left(X_n=0\right)=1-p$ ,当 $n$ 趋近于无穷大时,样本均值 $\bar{X}_n=\frac{1}{n} \sum_{i=1}^n X_i$ 收敛于 $p$ ,即:
		$$
		\lim _{n \rightarrow \infty} P\left( |\bar{X}_n-p|<\epsilon\right)=1.
		$$
	\end{thm}
	\begin{proof}
		伯努利大数定理的证明可以基于切比雪夫不等式。考虑到 $X_n$ 是伯努利随机变量,我们有 $E\left[X_n\right]=p$ 和 $\operatorname{Var}\left[X_n\right]=p(1-p)$. 样本均值 $\bar{X}_n$ 的期望和方差分别为:
		$$
		\begin{gathered}
			E\left[\bar{X}_n\right]=E\left[\frac{1}{n} \sum_{i=1}^n X_i\right]=p, \\
			\operatorname{Var}\left[\bar{X}_n\right]=\operatorname{Var}\left[\frac{1}{n} \sum_{i=1}^n X_i\right]=\frac{p(1-p)}{n},
		\end{gathered}
		$$
		根据切比雪夫不等式,我们有:
		$$
		P\left(\left|\bar{X}_n-p\right|>\epsilon\right) \leq \frac{\operatorname{Var}\left[\bar{X}_n\right]}{\epsilon^2}=\frac{p(1-p)}{n \epsilon^2},
		$$
		当 $n$ 趋于无穷大时,上式的右边趋于 0 ,因此
		$$
		\lim _{n \rightarrow \infty} P\left(\left|\bar{X}_n-p\right|>\epsilon\right)=0,
		$$
		这就证明了样本均值 $\bar{X}_n$ 依概率收敛于 $p$.
	\end{proof}
 
 	
 	\begin{thm}[辛钦大数定理 (Khintchine's Law of Large Numbers)]
 		对于一组独立同分布的随机变量 $\left\{X_n\right\}$ ,如果它们的期望 $E\left[X_n\right]$ 存在,则样本均值 $\bar{X}_n=$ $\frac{1}{n} \sum_{i=1}^n X_i$ 当 $n$ 趋近于无穷大时依概率收敛于 $E\left[X_n\right]$ ,即对于任意 $\epsilon>0$ :
 		$$
 		\lim _{n \rightarrow \infty} P\left(\left|\bar{X}_n-E\left[X_n\right]\right|>\epsilon\right)=0.
 		$$
 	\end{thm}
 	\begin{proof}
 		辛钦大数定理的证明通常基于特征函数或者矩生成函数. 这里我们给出一个基于特征函数的简化证明.
 		
 		由于 $\left\{X_n\right\}$ 是独立同分布的,并且 $E\left[X_n\right]$ 存在,我们可以定义 $Y_n=X_n-E\left[X_n\right]$ 使得 $E\left[Y_n\right]=0$. 对于任意实数 $t$ ,特征函数 $\phi_{Y_n}(t)$ 存在,并且我们有:
 		$$
 		\phi_{\bar{Y}_n}(t)=\left[\phi_{Y_n}\left(\frac{t}{n}\right)\right]^n
 		,$$
 		由于 $E\left[Y_n\right]=0$ ,我们可以将 $\phi_{Y_n}(t)$ 在 $t=0$ 处泰勒展开:
 		$$
 		\phi_{Y_n}(t)=1+i t E\left[Y_n\right]-\frac{t^2}{2} E\left[Y_n^2\right]+o\left(t^2\right),
 		$$
 		这里 $o\left(t^2\right)$ 表示当 $t$ 趋近于0时比 $t^2$ 更快地趋近于0的项.将上述展开带入 $\phi_{Y_n}(t)$ 的表达式得到:
 		$$
 		\phi_{Y_n}(t)=\left[1-\frac{t^2}{2 n^2} E\left[Y_n^2\right]+o\left(\frac{t^2}{n^2}\right)\right]^n,
 		$$
 		当 $n$ 趋近于无穷大时,由于 $E\left[Y_n^2\right]<\infty$ ,我们有:
 		$$
 		\lim _{n \rightarrow \infty} \phi_{Y_n}(t)=e^{-\frac{t^2}{2} E\left[Y_n^2\right]},
 		$$
 		上述极限就是均值为 0 、方差为 $E\left[Y_n^2\right]$ 的正态分布的特征函数.由Lévy极限定理, $\bar{Y}_n$ 依分布收敛于均值为 0 、方差为 $E\left[Y_n^2\right]$ 的正态分布. 由于 $\bar{X}_n-E\left[X_n\right]=\bar{Y}_n$ ,我们得到 $\bar{X}_n$ 依 分布收敛于均值为 $E\left[X_n\right]$ 、方差为 $\mathrm{O}$ 的常量分布,即 $\bar{X}_n$ 依概率收敛于 $E\left[X_n\right]$.
 	\end{proof}
 
 
 	\begin{thm}[切比雪夫大数定理 (Chebyshev's Law of Large Numbers)]
 		如果一个随机变量序列 $\left\{X_n\right\}$ 独立同分布,且存在常数 $E\left[X_n\right]=\mu$ 和 $\operatorname{Var}\left[X_n\right]=\sigma^2<$ $\infty$ ,那么对于任意 $\epsilon>0$ ,有
 		$$
 		\lim _{n \rightarrow \infty} P\left(\left|\frac{1}{n} \sum_{i=1}^n X_i-\mu\right| \geq \epsilon\right)=0
 		$$.
 	\end{thm}
	\begin{proof}
		这个定理的证明可以直接使用切比雪夫不等式. 对于任意 $\epsilon>0$ ,我们有
		$$
		P\left(\left|\frac{1}{n} \sum_{i=1}^n X_i-\mu\right| \geq \epsilon\right) \leq \frac{\operatorname{Var}\left[\frac{1}{n} \sum_{i=1}^n X_i\right]}{\epsilon^2}=\frac{\sigma^2}{n \epsilon^2},
		$$
		由于 $\sigma^2$ 是有限的,当 $n$ 趋于无穷时,上式右侧趋于 0. 因此,
		$$
		\lim _{n \rightarrow \infty} P\left(\left|\frac{1}{n} \sum_{i=1}^n X_i-\mu\right| \geq \epsilon\right)=0.
		$$
	\end{proof}
	
	\begin{thm}[马尔科夫大数定理 (Markov's Law of Large Numbers)]
		如果一个随机变量序列 $\left\{X_n\right\}$ 独立同分布,且存在常数 $E\left[X_n\right]=\mu$ ,那么对于任意 $\epsilon>0$,有
		$$
		\lim _{n \rightarrow \infty} P\left(\left|\frac{1}{n} \sum_{i=1}^n X_i-\mu\right| \geq \epsilon\right)=0
		.$$
	\end{thm}
	
	\begin{proof}
		我们定义 $S_n=X_1+X_2+\ldots+X_n$ 和 $\bar{X}_n=\frac{S_n}{n}$ 。由于 $X_1, X_2, \ldots, X_n$ 是独立同分布的,我们有 $E\left[X_i\right]=\mu$ 和 $\operatorname{Var}\left[X_i\right]=\sigma^2$ (我们假设方差存在且有限).
		我们想要证明的是
		$$
		\lim _{n \rightarrow \infty} P\left(\left|\bar{X}_n-\mu\right| \geq \epsilon\right)=0,
		$$
		根据切比雪夫不等式,我们有
		$$
		P\left(\left|\bar{X}_n-\mu\right| \geq \epsilon\right) \leq \frac{\operatorname{Var}\left[\bar{X}_n\right]}{\epsilon^2},
		$$
		由于 $\operatorname{Var}\left[\bar{X}_n\right]=\frac{\sigma^2}{n}$ (这个结果来自方差的性质和随机变量的独立性),我们可以进一步得 到
		$$
		P\left(\left|\bar{X}_n-\mu\right| \geq \epsilon\right) \leq \frac{\sigma^2}{n \epsilon^2},
		$$
		当 $n$ 趋于无穷大时,上式的右侧趋于 0 ,因此我们得到
		$$
		\lim _{n \rightarrow \infty} P\left(\left|\bar{X}_n-\mu\right| \geq \epsilon\right)=0.
		$$
	\end{proof}
	
\end{document}