\documentclass[UTF8,12pt]{ctexart}
%加载包
\usepackage{geometry}%排版
%a4版面,页边距1英寸,showframe 增加边框的参数。
% 设置为A4纸,边距适中模式(永中office)
\geometry{%
	width = 210mm,%
	height = 297mm,
	left = 19.1mm,%
	right = 19.1mm,%
	top = 25.4mm,%
	bottom = 25.4mm%
}
%\hyphenpenalty = 1000% 断字设置,值越大,断字越少。
%\setlength{\parindent}{2em}% 缩进
%\setlength{\parskip}{0.5ex} % 段间距

\usepackage{amsmath}
\usepackage{amsfonts}
\usepackage{amssymb}%公式
\usepackage{amsthm}%定理环境

%\usepackage{ntheorem}%定理环境,使用这个会使\maketitle版式出问题
\usepackage{bm}%加粗

\usepackage{mathrsfs}
\numberwithin{equation}{section}%对公式以节{section}为基础进行编号.变成(1.1.1)有chapter才有1.1.1,不然只有section是1.1
%\theoremstyle{plain}%定理用latex默认的版式
\newtheorem{thm}{Theorem}[section]
%\theoremstyle{definition}%定义用definition格式
\newtheorem{defn}{Definition}
%\theoremstyle{remark}%用remark格式
\newtheorem{lemma}[thm]{lemma}
\newtheorem{example}{Example}[section]

\usepackage{multirow}%表格列合并宏包,\multirow命令.

\usepackage{tabularx}%表格等宽,\begin{tabularx}命令.

\usepackage{tcolorbox}%盒子
\tcbuselibrary{skins, breakable}% 支持文本框跨页
\usepackage[english]{babel}% 载入美式英语断字模板

\usepackage{graphicx}

\usepackage{float}

\usepackage{listings}   %代码块
\usepackage{xcolor}
\definecolor{codegreen}{rgb}{0,0.6,0}
\definecolor{codegray}{rgb}{0.5,0.5,0.5}
\definecolor{codepurple}{rgb}{0.58,0,0.82}
\definecolor{backcolour}{rgb}{0.95,0.95,0.92}
%设置代码块
\lstdefinestyle{mystyle}{
	backgroundcolor=\color{backcolour},   
	commentstyle=\color{codegreen},
	keywordstyle=\color{magenta},
	numberstyle=\tiny\color{codegray},
	stringstyle=\color{codepurple},
	basicstyle=\ttfamily\footnotesize,
	breakatwhitespace=false,         
	breaklines=true,                 
	captionpos=b,                    
	keepspaces=true,                 
	numbers=left,                    
	numbersep=5pt,                  
	showspaces=false,                
	showstringspaces=false,
	showtabs=false,                  
	tabsize=2
}

\lstset{style=mystyle,
	language=R,                                       % 设置语言
}

\usepackage{appendix}%附录


%标题页
\title{Toy Matrix gradient}
\author{Renhe W.}
\date{ }
\usepackage{hyperref}%可以生成pdf书签,可以跳转
\hypersetup{
	colorlinks=true,
	linkcolor=black
}%使得目录没有红框

%工具
%使用文本框
%\begin{tcolorbox}[enhanced]	\end{tcolorbox}
%代码框
%{\setmainfont{Courier New Bold}                       %设置代码字体                   
%\begin{lstlisting}
	
%\end{lstlisting}}





%文章开始部分
\begin{document}
	\maketitle
	\tableofcontents
	\newpage
	
	%引言
	%附录
	\section{矩阵梯度\label{G}}
	\subsection{实值函数相对于实向量的梯度}
	对于$n\times 1$向量$\bm{X}$的梯度算子记作$\nabla_{\bm{X}}$,定义为:
		\begin{equation}
			\nabla_{\bm{X}}=[\frac{\partial }{\partial x_1},\frac{\partial }{\partial x_2},\cdots,\frac{\partial }{\partial x_n}]^T=\frac{\partial }{\partial\bm{X} }.
		\end{equation}
	\indent 因此,$n \times 1$实向量$\bm{X}$为变元的实际标量函数$f(\bm{X})$相对于$\bm{X}$的梯度是一个$n \times 1$的一列向量,定义为:
		\begin{equation}
			\nabla_{\bm{X}}f(\bm{X})=
			\begin{bmatrix}
				\frac{\partial f(\bm{X}) }{\partial x_1},&\frac{\partial f(\bm{X})}{\partial x_2},&\cdots, &\frac{\partial f(\bm{X})}{\partial x_n}
			\end{bmatrix}^T=
			\begin{bmatrix}
				\frac{\partial f(\bm{X}) }{\partial x_1}\\
				\frac{\partial f(\bm{X})}{\partial x_2}\\
				\vdots \\
			\frac{\partial f(\bm{X}) }{\partial x_n}	
			\end{bmatrix}
		=\frac{\partial f(\bm{X})}{\partial\bm{X}}.
		\end{equation}
	\indent 梯度方向的负方向为变元$\bm{X}$的梯度流(gradient flow),记作:
	\begin{equation}
		\dot{\bm{X}}=-\nabla_{\bm{X}}f(\bm{X}).
	\end{equation}
	\indent 可以看出:
	\begin{enumerate}
		\item 一个以向量为变元函数的梯度为一个向量;
		\item 梯度的每个分量代表变量函数在该分量方向上的变化率.
	\end{enumerate}
	\begin{tcolorbox}[enhanced]	
	\quad 梯度向量最重要的性质之一是,它指出了当变元增大时函数$f$的最大增大 率。相反,梯度的负值(负梯度)指出了当变元增大时函数$f$的最大减小率。根 据这样一种性质,即可设计出求一函数极小值的迭代算法。
	\end{tcolorbox}
	
	\indent 类似地,实值函数$f(x)$相对于$1 \times n$一行向量$x^T$的梯度为$1 × n$一行行向量,定义为:
		\begin{equation}
		\nabla_{\bm{X}^T}f(\bm{X})=
		\begin{bmatrix}
		\frac{\partial f(\bm{X}) }{\partial x_1},&\frac{\partial f(\bm{X})}{\partial x_2}  ,&\cdots  ,&\frac{\partial f(\bm{X}) }{\partial x_n}
		\end{bmatrix}
		=\frac{\partial f(\bm{X})}{\partial\bm{X}^T}.
	\end{equation}
	\indent $m$维一行向量函数$f(\bm{X})=[f_1(\bm{X}),\cdots,f_m(\bm{X}) ]$对$n$维一列向量$\bm{X}$的梯度为$ n\times m$矩阵\footnote{$f(x)$为列,$X$为行.}:
		\begin{equation}\label{eq:函数对向量梯度}
		\begin{bmatrix}
			\frac{\partial f_1(\bm{X}) }{\partial x_1}	&	\frac{\partial f_2(\bm{X}) }{\partial x_1}  & \cdots &	\frac{\partial f_n(\bm{X}) }{\partial x_1} \\
			\frac{\partial f_1(\bm{X}) }{\partial x_2}	&	\frac{\partial f_2(\bm{X}) }{\partial x_2}  & \cdots &	\frac{\partial f_n(\bm{X}) }{\partial x_2} \\
			\vdots& \vdots & \cdots & \vdots\\
		\frac{\partial f_1(\bm{X}) }{\partial x_n}	&	
		\frac{\partial f_2(\bm{X}) }{\partial x_n}  & \cdots &	\frac{\partial f_n(\bm{X}) }{\partial x_n} \\
		\end{bmatrix}.
		\end{equation}
		\indent 下面举几个例子:
		\begin{example}
			若$f(x)=[x_1,\cdots,x_n]$,则$\frac{\partial \bm{X}^T }{\partial \bm{X}}=\bm{I}$:\\\indent
			由(\ref{eq:函数对向量梯度}),有:
			$$\begin{bmatrix}
				\frac{\partial x_1 }{\partial x_1}	&	\frac{\partial x_2 }{\partial x_1}  & \cdots &	\frac{\partial x_n }{\partial x_1} \\
				\frac{\partial x_1 }{\partial x_2}	&	\frac{\partial x_2}{\partial x_2}  & \cdots &	\frac{\partial x_n }{\partial x_2} \\
				\vdots& \vdots & \cdots & \vdots\\
				\frac{\partial x_1 }{\partial x_n}	&	
				\frac{\partial x_2 }{\partial x_n}  & \cdots &	\frac{\partial x_n }{\partial x_n} \\
			\end{bmatrix}=
		\begin{bmatrix}
			1	&	0  & \cdots &0 \\
			0	&	1 & \cdots & 0 \\
			\vdots& \vdots & \cdots & \vdots\\
			0	&	0  & \cdots &1 \\
		\end{bmatrix}=\bm{I}.
		$$
			
		\end{example}
		
		\subsection{实值函数的梯度矩阵}
		在最优化问题中,需要最优化的对象可能是某个加权矩阵。因此,有必要 分析实值函数相对于矩阵变元的梯度。实值函数$f(A)$相对于$m\times n$是矩阵$A$的梯度为一$m\times n$矩阵,简称{\color{brown}梯度矩阵}, 定义为:
		\begin{equation}\label{eq:实值函数的梯度矩阵}
		\frac{\partial f(\bm{A}) }{\partial \bm{A}}	=
			\begin{bmatrix}
				\frac{\partial f(\bm{A}) }{\partial A_{11}}	&	\frac{\partial f(\bm{A}) }{\partial A_{12}} & 
				\cdots &
				\frac{\partial f(\bm{A}) }{\partial A_{1n}} \\
				\frac{\partial f(\bm{A}) }{\partial A_{21}}	&	
				\frac{\partial f(\bm{A}) }{\partial A_{22}} &
				 \cdots & 
				 \frac{\partial f(\bm{A}) }{\partial A_{2n}} \\
				\vdots& \vdots & \cdots & \vdots\\
				\frac{\partial f(\bm{A}) }{\partial A_{m1}}	&
				\frac{\partial f(\bm{A}) }{\partial A_{m2}} & 
				\cdots &
				\frac{\partial f(\bm{A}) }{\partial A_{mn}}\\
			\end{bmatrix}.
		\end{equation}
		\indent 式中$A_{ij}$为矩阵$A$的元素.
		\\\indent 实值函数相对于矩阵变元的梯度具有以下性质:
		\begin{example}
			若$f(\bm{A})=c$是常数,其中$\bm{A}$为$m\times n$矩阵,则梯度$\frac{\partial c }{\partial A}=\bm{O}_{m\times n}$.
		\end{example}
		\begin{example}
			若$\bm{A}\in R^{m\times n}$,$x\in R^{m\times 1}$,$y\in R^{n\times 1}$,则$\frac{\partial x^T\bm{A}y }{\partial \bm{A}}=xy^T$.
			\\\indent ANS:\\\indent
			$$\bm{A}=
			\begin{bmatrix}
				A_{11}	&A_{12} & \cdots & A_{1n} \\
				A_{21}	&A_{22} & \cdots & A_{2n} \\
				\vdots& \vdots & \cdots & \vdots\\
				A_{m1}	&A_{m2}  & \cdots &A_{mn} \\
			\end{bmatrix},
			x=\begin{bmatrix}
				x_1\\
				x_2\\
				\vdots\\
				x_m
			\end{bmatrix},
			y=\begin{bmatrix}
				y_1\\
				y_2\\
				\vdots\\
				y_m
			\end{bmatrix}.
			$$
		\indent	$\frac{\partial x^T\bm{A}y }{\partial \bm{A}}=\frac{\partial \sum_{i=1}^{m}\sum_{j=1}^{n}x_iA_{ij}y_j }{\partial \bm{A}}$,由(\ref{eq:实值函数的梯度矩阵}),$\frac{\partial \sum_{i=1}^{m}\sum_{j=1}^{n}x_iA_{ij}y_j }{\partial \bm{A}}=
			\begin{bmatrix}
			x_1y_1	&x_1y_2 & \cdots & x_1y_n \\
			x_2y_1	&x_2y_2 & \cdots & x_2y_n \\
			\vdots& \vdots & \cdots & \vdots\\
			x_my_1	&x_my_2  & \cdots &x_my_n \\
		\end{bmatrix}=xy^T.
		$
		\end{example}

	
\end{document}